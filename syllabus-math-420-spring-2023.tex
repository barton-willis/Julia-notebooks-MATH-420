\documentclass[12pt,fullpage]{article}
\usepackage{multicol}
\usepackage[super]{nth}
\usepackage{wasysym}
\usepackage{phonenumbers}
\usepackage{marvosym }
\usepackage{xcolor}
\usepackage{comment}
\usepackage{pdfpages}
\usepackage[paperwidth=8.5in,paperheight=11in,margin=0.75in]{geometry} 
\usepackage[UKenglish]{babel}
\usepackage[UKenglish]{isodate}% http://ctan.org/pkg/isodate
\usepackage[colorlinks=true,linkcolor=black,anchorcolor=black,citecolor=black,filecolor=black,menucolor=black,runcolor=black,urlcolor=black]{hyperref}
\usepackage[activate={true,nocompatibility},final,tracking=true,kerning=true,spacing=true,factor=1100,stretch=10,shrink=10]{microtype}
\frenchspacing
\usepackage[nodayofweek,level]{datetime}
\usepackage{calc,url}
\newcounter{qz}\setcounter{qz}{0}
\newcommand{\qz}{\setcounter{qz}{\value{qz}+1}\textbf{ \theqz}}

\newcounter{hw}\setcounter{hw}{0}
\newcommand{\hw}{%\
\setcounter{hw}{\value{hw}+1}
\textbf{HW \thehw}}

\newcounter{ex}\setcounter{ex}{0}
\newcommand{\ex}{%\
\setcounter{ex}{\value{ex}+1}
Exam \theex}

\usepackage{fourier}
\usepackage[T1]{fontenc}

\newenvironment{mypar}[2]
  {\begin{list}{}%
    {\setlength\leftmargin{#1}
    \setlength\rightmargin{#2}}
    \item[]}
  {\end{list}}


\newcounter{wk}\setcounter{wk}{0}
\newcommand{\wk}{%\
\setcounter{wk}{\value{wk}+1}
\thewk \,\,}

\usepackage[nomessages]{fp}% http://ctan.org/pkg/fp


\usepackage{enumerate}
\usepackage{graphicx}

\usepackage{paralist}
\renewenvironment{description}[0]{\begin{compactdesc}}{\end{compactdesc}}

\newenvironment{alphalist}{
  \begin{enumerate}[(a)]
    \addtolength{\itemsep}{-0.5\itemsep}}
  {\end{enumerate}}
  \cleanlookdateon% Remove ordinal day reference
  \newcommand{\RomanNumeralCaps}[1]
      {\MakeUppercase{\romannumeral #1}}

\usepackage{xspace}
\makeatletter
\DeclareRobustCommand{\maybefakesc}[1]{%
  \ifnum\pdfstrcmp{\f@series}{\bfdefault}=\z@
    {\fontsize{\dimexpr0.8\dimexpr\f@size pt\relax}{0}\selectfont\uppercase{#1}}%
  \else
    \textsc{#1}%
  \fi
}
\newcommand\AM{\,\maybefakesc{am}\xspace}
\newcommand\PM{\,\maybefakesc{pm}\xspace}
\makeatother

\newcommand{\coursename}{Numerical Analysis}
\newcommand{\coursenumber}{MATH 420}
\newcommand{\sectionnumber}{01}
\newcommand{\term}{Spring }
\newcommand{\room}{Discovery Hall, room  383}
\newcommand{\meetingtime}{This class meets Monday, Wednesday, and Friday  from 
	9:05\PM{}  --  9:55 \PM}
\newcommand{\ay}{2023}
\newcommand{\finaldateandtime}{Wednesday 17, May  from 8:00\AM{}--10:00\AM}
\newcommand{\finalexam}{The final exam will be comprehensive and it 
will be given on Wednesday, 17 May from 8:00\AM{}--10:00\AM}  
\newcommand{\officehours}{ Monday, Wednesday, and Friday \mbox{10:00\AM{} -- 11:00\AM},
    Tuesday and Thursday \mbox{9:30\AM -- 11:00\AM}, and by appointment.}
\begin{document}
\cleanlookdateon% Remove ordinal day reference
\shortdate
\printyearoff
\large
\begin{center}
    \textbf{\coursename}  \\
    {\coursenumber--\sectionnumber} \\
     {\term \ay} \\
\end{center}

\vskip0.25in
\normalsize

\begin{center}
    \begin{description}
        \item[Instructor:] Barton Willis, PhD, Professor of Mathematics
        \item[Office:]  Discovery Hall, Room 368
        \item[\phone:]   \phonenumber[country=US]{3088658868}
        \item[\Email:]    \href{mailto:willisb@unk.edu}{willisb@unk.edu}
        %\item[Zoom:] 616 568 5706
        \item[Office Hours:] \officehours
      \end{description}
    \end{center}

    \subsubsection*{Important Dates}

\begin{mypar}{0.25in}{0.25in} 

      \textbf{First Homework due} \dotfill  \textbf{\printdate{4/2/\the\year}}  \\
       \textbf{Exam 1} \dotfill \textbf{\printdate{24/2/\the\year}}  \\
    \textbf{Exam 2} \dotfill  \textbf{\printdate{3/4/\the\year}} \\
    \textbf{Exam 3} \dotfill \textbf{\printdate{5/5/\the\year}} \\
         \textbf{Final exam} \dotfill  \textbf{\printdate{17/5/\the\year}}
\end{mypar}
    \subsubsection*{Class meeting times}

\meetingtime{} in \room.



\subsubsection*{Course Objectives}

On completion of this course, students will
\begin{alphalist}
    \item understand IEEE arithmetic and know the rules for accurate computation.
    \item understand the concepts of linear and quadratic convergence and use these concepts to analyze 
        the efficiency of an algorithm.
    \item develop an understanding of the algorithms for solving linear and nonlinear equations, interpolation, 
       quadrature, least squares methods, and solution of differential equations.
    \item be able to use a programming language and graphical tools to solve problems 
    numerically.
\end{alphalist}

\subsubsection*{Catalog description}

    \textbf{MATH 420} (Numerical Analysis, 3 credit hours) Principles of
    error analysis and accurate computation; rates of convergence, the solution 
    of linear and nonlinear equations, interpolation and least squares, 
    numerical integration, and numerical solution of differential equations.


\subsubsection*{Prerequisite}

To be in this class, you must have already earned a passing grade in Calculus II (UNK's MATH 202).

\subsubsection*{Final Exam}

\finalexam{} in \room.

\subsubsection*{Course Resources}

\begin{enumerate}

\item Our textbook is \emph{First Semester in Numerical Analysis with Julia,} by Giray Ökten.  This is an open-source 
textbook that can be legally downloaded, printed, and  used without 
payment.\footnote{ \url{https://open.umn.edu/opentextbooks/textbooks/first-semester-in-numerical-analysis-with-julia}.  \normalsize} 

\item A free account on UNL's supercomputer.

\item Class notes that are written on the board or distributed via Canvas.

\item Reliable internet access.

\item  An internet connected computer (not just a phone or tablet). 

\item The Julia programming language, the Jupyter notebook, and the Gadfly
      graphics package.

\item Pencils, erasers, notebook for note taking. Colored pens or pencils are nice 
for note taking.


 \end{enumerate}


\subsubsection*{Grading}
 Your course grade will be based on homework, three midterm exams, and a 
 comprehensive final exam; specifically:
 \begin{mypar}{0.25in}{0.25in}
       \textbf{Homework}  \emph{11 fifteen point assignments}  \dotfill 165 (total) \\
       \textbf{Mid-term Exams 1,2, and 3} \emph{100 points each} \dotfill 300 (total)\\
          \textbf{Comprehensive Final exam} \dotfill 150 (total)
 \end{mypar}

 
 \FPeval{\points}{round(165+300+150,0)}
 
 \FPeval{\F}{round(\points*0.6-1,0)}
 \FPeval{\Dm}{round(\points*0.6,0)}
 \FPeval{\D}{round(\points*0.633,0)}
 \FPeval{\Dp}{round(\points*0.6667,0)}
 
 \FPeval{\Cm}{round(\points*0.7,0)}
 \FPeval{\C}{round(\points*0.733,0)}
 \FPeval{\Cp}{round(\points*0.7667,0)}
 
 \FPeval{\Bm}{round(\points*0.8,0)}
 \FPeval{\B}{round(\points*0.833,0)}
 \FPeval{\Bp}{round(\points*0.8667,0)}
 
 \FPeval{\Am}{round(\points*0.9,0)}
 \FPeval{\A}{round(\points*0.933,0)}
 \FPeval{\Ap}{round(\points*0.98,0)}
\noindent The following table shows the \emph{minimum} number of points (out of \points) that
 are required for each of the twelve letter grades D- through A+. For
 example, a point total of \Bp\/  points will earn you a grade of B+,  and 
 a point total of \Am\/ points will earn you a grade of A-. If you earn a point
 total of \F\/  or less, you will earn a failing course grade.
  
  \vspace{0.1in}
      \begin{minipage}{5.5in}
   \centering 
 \begin{mypar}{0.25in}{0.25in}
     \begin{minipage}{2.5in}
         D-  \dotfill \Dm \\
         D \dotfill \D \\
         D+ \dotfill \Dp \\
         C- \dotfill \Cm  \\
         C \dotfill \C \\
         C+ \dotfill \Cp 
         \end{minipage}
     \phantom{xxx}
     \begin{minipage}{2.5in}
         B- \dotfill \Bm \\
         B \dotfill  \B \\
         B+ \dotfill  \Bp\\
         A- \dotfill  \Am \\
         A \dotfill  \A \\
         A+ \dotfill  \Ap
     \end{minipage}
 \end{mypar} 
 \end{minipage}

 \subsubsection*{Course Calendar}

 We will try to adhere to the following schedule, but we will modify it
 if needed. The exam dates will only be changed for a compelling
 reason; we won't delay an exam because we are behind the
 schedule. Neither will an exam date be moved forward because we are
 ahead of the schedule.
 
 Homework assignments are due at 23:59 local time on Saturday on the week they are assigned. 
 
 \vspace{0.1in}
 
 \begin{center}
 
 \begin{tabular}  {|l|l|l|l|}
 \hline
 {\bf Week}  & {\bf Week} &  {\bf Section(s)} & {\bf Topic(s)} \\
 \hline \hline 
 \wk    & 1/23 &    \S1.2   &   Introduction to Julia   \hfill \\
 \wk    & 1/30 & \S1.1      &  Floating point numbers \& calculus tools   \hfill \textbf{HW  \qz,  HW \qz} \\
 \wk    & 2/6 &    \S2.1 & Errors and convergence rates    \hfill \textbf{HW  \qz} \\
 \wk    & 2/13  & \S2.2 -- \S2.7 &  Root Finding   \hfill  \textbf{HW \qz} \\
 \wk    & 2/20 &  \S2.2--\S2.7   &  Root Finding    \hfill \textbf{\ex\-\-, 24 February}   \\
 \wk    & 2/27    &  &  Linear equations  \hfill  \textbf{HW  \qz }  \\
 \wk    & 3/6    &  &  Linear equations  \hfill  \textbf{HW  \qz }  \\
 \wk    & 3/20     & \S3.1--\S3.4 & Interpolation \hfill \textbf{ HW  \qz }   \\
 \wk    & 3/27   & \S3.1 -- \S3.4   &   Interpolation   \hfill \textbf{ HW  \qz} \\
 \wk    & 4/3   & \S4.1--\S4.2 &  Numerical  integration  \hfill \textbf{\ex\-\-, 7 April} \\
 \wk    & 4/10    &   \S4.3 --\S4.4 &   Numerical integration    \hfill  \textbf{ HW  \qz} \\
 \wk    & 4/17 &   \S5.1--\S5.2 & Discrete \& continuous least squares  \hfill \textbf{HW    \qz} \\
 \wk    & 4/24  & \S5.3 &  Orthogonal polynomials and least squares  \hfill \textbf{HW \qz}   \\
 \wk    & 5/1  &   & Differential equations \hfill  \textbf{ \ex\-\-\-, 5 May}     \\
 \wk    & 5/8  &     & Differential equations   \hfill \\
 \wk    & 5/15       &  &   \textbf{Final Exam, \finaldateandtime} \\ \hline
 \end{tabular}
 \end{center}
 


\subsubsection* {Policies}


\begin{enumerate}

\item All work you turn in for a grade must be your own.  If you need assistance in completing a homework assignment, you may ask me for help but nobody else. Googling for
answers, seeking help from the Learning Commons or other faculty members,  or using solution keys from previous terms (either from UNK or other universities) is also prohibited.  Each homework assignment you turn in for a grade must include the statement:

\begin{quote}
\fbox{I have neither given nor received unauthorized assistance on this assignment.}
\end{quote}
Using unauthorized materials while taking a test will earn you a failing course grade. Each exam will specify what resources are allowed. 

 \item The final examination will be \emph{comprehensive}. It will be given on Wednesday 5 May from 8:00 am to 10:00 am.


 \item Homework is due at 11:59 \PM local time on Saturday. If you have an 
 extended illness that keeps you from completing the homework, contact
  me immediately. Since homework is turned in electronically, requests to 
turn in homework late (due to minor illness or absences) will generally be declined.

\item Please turn in your homework on time. Homework that is turned in after I start to grade the
papers will \emph{always} earn a grade of zero. 

\item The course calendar may change. Changes to the schedule can be made in class, but 
not noted anywhere else. It is your responsibility to
learn of changes to the schedule.



\item If you have questions about how your work has been graded,  \emph{immediately}  ask me for an explanation.

\item This class has \emph{no option for extra credit.}


\item  For pedagogical reasons, our class notes will sometimes differ in notation, 
style, and level of abstraction from the textbook.


\end{enumerate}




\subsubsection*{Students with Disabilities or Those Who are Pregnant}

It is the policy of the University of Nebraska at Kearney to provide flexible and individualized reasonable accommodation to students with documented disabilities. To receive accommodation services for a disability, students must be registered with UNK Disabilities Services for Students Office, 172 Memorial Student Affairs Building, 308-865-8988 or by email unkdso@unk.edu


It is the policy of the University of Nebraska at Kearney to provide flexible and individualized reasonable accommodation to students who are pregnant. To receive accommodation services due to pregnancy, students must contact Cindy Ference in Student Health, 308-865-8219. The following link provides information for students and faculty regarding pregnancy rights:  \small \url{ http://www.nwlc.org/resource/pregnant-and-parenting-students-rights-faqs-college-and-graduate-students} \normalsize 

\subsubsection*{Reporting Student Sexual Harassment, Sexual Violence or Sexual Assault}


Reporting allegations of rape, domestic violence, dating violence, sexual assault, sexual harassment, and stalking enables the University to promptly provide support to the impacted student(s), and to take appropriate action to prevent a recurrence of such sexual misconduct and protect the campus community. Confidentiality will be respected to the greatest degree possible. Any student who believes she or he may be the victim of sexual misconduct is encouraged to report to one or more of the following resources:

\begin{enumerate}

\item Local Domestic Violence, Sexual Assault Advocacy Agency 308-237-2599

\item Campus Police (or Security) 308-865-8911

\item Title IX Coordinator 308-865-8655

Retaliation against the student making the report, whether by students or University employees, will not be tolerated.

\end{enumerate}

\subsubsection*{Academic Honesty Policy\footnote{\url{https://catalog.unk.edu/undergraduate/academics/academic-regulations/academic-integrity-policy/}}}

Academic honesty is essential to the 
existence and integrity of an institution of higher education. The
responsibility for maintaining that integrity is shared by all members of the academic community. To
further serve this end, the University of Nebraska at Kearney has a policy relating to academic integrity.
The maintenance of academic honesty and integrity is a vital concern of the University community. Any
student found in violation of the standards of academic integrity may be subject to both academic and
disciplinary sanctions. Academic dishonesty includes, but is not limited to, the following:

\paragraph{Cheating:} Copying or attempting to copy from an academic test or examination of another student; using
or attempting to use unauthorized materials, information, notes, study aids or other devices for an academic
test, examination or exercise; engaging or attempting to engage the assistance of another individual in
misrepresenting the academic performance of a student; or communicating information in an unauthorized
manner to another person for an academic test, examination or exercise.

\paragraph{Fabrication and falsification:} Falsifying or fabricating any information or citation in any academic exercise,
work, speech, test or examination. Falsification is the alteration of information, while fabrication is the
invention or counterfeiting of information.

\paragraph{Plagiarism:} Presenting the work of another as one's own (i.e., without proper acknowledgment of the source)
and submitting examinations, theses, reports, speeches, drawings, laboratory notes or other academic work
in whole or in part as one's own when such work has been prepared by another person or copied from another
person.

\paragraph{Abuse of academic materials and/or equipment:} Destroying, defacing, stealing, or making inaccessible library
or other academic resource material.

\paragraph{Complicity in academic dishonesty:} Helping or attempting to help another student to commit an act of
academic dishonesty.

\paragraph{Falsifying grade reports:} Changing or destroying grades, scores or markings on an examination or in an
instructor's records.

\paragraph{Misrepresentation to avoid academic work:} Misrepresentation by fabricating an otherwise justifiable excuse
such as illness, injury, accident, etc., in order to avoid or delay timely submission of academic work or to
avoid or delay the taking of a test or examination.

\paragraph{Other Acts of Academic Dishonesty:} Academic units and members of the faculty may prescribe and give
students prior written notice of additional standards of conduct for academic honesty in a particular course,
and violation of any such standard shall constitute a violation of the Code.

Under \S 2.9 of the Bylaws of the Board of Regents of the University of Nebraska, the respective colleges
of the University have responsibility for addressing student conduct solely affecting the college. Just as the
task of inculcating values of academic honesty resides with the faculty, the college faculty are entrusted with
the discretionary authority to decide how incidents of academic dishonesty are to be resolved. For more
information, please visit UNK's Procedures and Sanctions for Academic Integrity and the Student Code of
Conduct.












\newpage





\end{document}

\vspace{0.1in}

\begin{center}

\begin{tabular}  {|l|r|l|l|}
\hline
{\bf Week}  & {\bf Week of} & {\bf Month}&  {\bf Topic(s)} \\
\hline \hline
\wk    & 12 & Jan  & algorithms and an introduction to Maxima (classnotes) \\\
\wk    & 19 &     &  IEEE arithmetic and the condition number (Chapter 0) \\
3    & 26 &     & interval arithmetic and running errors (classnotes) \\
4    & \phantom{1}2 & Feb   &  solving equations, Chapter 1 \\
5    & 9  &  & solving equations, Chapter 1; \textbf{Exam 1, 13 Feb.} \\
6    & 16 & & numerical linear algebra, Chapter 2 \\
7    & 23 &  & numerical linear algebra, Chapter 2 \\
8    & \phantom{1}2 & Mar  & interpolation and least squares, Chapters 3 and 4 \\
9    & 9 &    & numerical differentiation and integration, Chapter 5; \textbf{Exam 2, 13 March} \\
10    & 23 &      & numerical differentiation and integration, Chapter 5 \\
11   & 30 &     & initial value problems, Chapter 6 \\
12   & 6  & April   &   initial value problems, Chapter 6 \\  
13   & 13 &        & partial differential equations, Chapter 8 \\
14   & 20 &     & partial differential equations, Chapter 8; \textbf{Exam 3, 24 April} \\
15   & 27 &     & Fourier series  (classnotes) \\ \hline
     &      &     & \textbf{Final Exam, Wednesday 6 May, 10:30--12:30} \\ \hline
\end{tabular}
\end{center}


\end{document}

\begin{tabular}  {|l|l|l|l|}
\hline
{\bf Week}  & {\bf Week of} & {\bf Month}&  {\bf Topic(s)} \\
\hline \hline
1    & \phantom{1}9 &Jan  & Introduction \\
2    & 16 &   & Floating point numbers \\
3    & 23 &  & Condition number \& relative difference \\
4    & 30 &  & Floating point arithmetic \\
5    & \phantom{1}6 &Feb & Sums \& hypergeometic functions; \textbf{Exam 1, Friday, 11 Feb.} \\
6    & 13 &    & Linear algebra \\
7    & 20 &     & Matrix norms and condition numbers \\
8    & 27 &     & LU factorization \\
9    & \phantom{1}6 & Mar      & Eigenvalues \\
10   & 20 &    & Least squares; \textbf{Exam 2, Friday, 25 March} \\
11   &   27   &  & Interpolation \& approximation \\
12   &   \phantom{1}3 & April   & Nonlinear equations \\
13   &   10 &     &  Numerical Integration \\
14   &   17 &     & Finite differences; \textbf{Exam 3, Friday, 22 April} \\
15   &   24 &     & Partial Differential Equations \\ \hline
     &      &     & \textbf{Final Exam, Wednesday, 3 May, 8:00--10:00} \\ \hline
\end{tabular}

3    & 24  & \S2.3---\S2.4  & Norms and condition numbers \\
4    & 31  & \S3.1---\S3.3  & Least squares \\
5    & 7 Feb & \S4.1           & Eigenvalues; {\bf Exam 1, Friday, 10 Feb}\\ \hline
6    & 14 Feb & \S4.2           & Eigenvalues  \\
7    & 21 & \S5.1---\S5.2  & Nonlinear equations \\
8    & 28  & \S7.1---\S7.2  & Interpolation \& Approximation \\
9    & 7 Mar  & \S8.1---\S8.3    & Numerical integration; {\bf Exam 2, Friday, 24 March}\\
\hline
10   & 21 Mar & \S8.4           & Adaptive Integration \\
11   & 28 & \S8.7, \S9.1---\S9.2  & Finite Differences and Initial Value Problems \\
12   & 4 Apr & \S9.3---\S9.4  & Differential Equations  \\
13   & 11  & \S11.1---\S11.2 & Partial Differential Equations \\
14   & 18   & \S11.2  &   Partial Differential Equations; {\bf Exam 3, Friday. 21 April} \\
15   &  & \S12.1---\S12.3  & Fast Fourier Transform \\ \hline
16    &  & \S1.1---\S12.3    &  {\bf Final Exam Wednesday 3 May 8:00--10:00}  \\
\hline
\end{tabular}
\end{center}
\end{document}


   1.  Scientific Computing
   2. Systems of Linear Equations
   3. Linear Least Squares
   4. Eigenvalue Problems
   5. Nonlinear Equations
   6. Optimization
   7. Interpolation
   8. Numerical Integration and Differentiation
   9. Initial Value Problems for Ordinary Differential Equations
  10. Boundary Value Problems for Ordinary Differential Equations
  11. Partial Differential Equations
  12. Fast Fourier Transform
  13. Random Numbers and Stochastic Simulation 










\begin{center}

\begin{tabular}  {|l|l|l|l|}
\hline
{\bf Week}  & {\bf Week of} & {\bf Month}&  {\bf Topic(s)} \\
\hline \hline
1    & 14 & Jan  & Floating point numbers, Chapter 0 \\
2    & 21 &     & The condition number \& IEEE arithmetic, classnotes \\
3    & 28 &     & Running errors \& IEEE arithmetic, classnotes \\
4    & \phantom{1}4 & Feb   &  Solving equations, Chapter 1 \\
5    & 11  &  & Solving equations, Chapter 1; \textbf{Exam 1, 15 Feb.} \\
6    & 18 & & Numerical linear algebra, Chapter 2 \\
7    & 25 &  & Numerical linear algebra, Chapter 2 \\
8    & \phantom{1}3 & Mar  & Interpolation and least squares, Chapters 3 and 4 \\
9    & 10 &    & Numerical differentiation and integration, Chapter 5 \\
10    & 24 &      & Numerical differentiation and integration, Chapter 5;   \textbf{Exam 2, 28 March} \\
11   & 31 &     & Initial value problems, Chapter 6 \\
12   & \phantom{1}7  & April  &   Initial value problems, Chapter 6 \\  
13   & 14 &       & Partial differential equations, Chapter 8 \\
14   & 21 &     & Partial differential equations, Chapter 8; \textbf{Exam 3, 25 April} \\
15   & 28 &     & Fourier series, classnotes \\ \hline
     &      &     & \textbf{Final Exam, Wednesday 6 May, 10:30--12:30} \\ \hline
\end{tabular}
\end{center}








\end{document}

\subsubsection*{Prerequisites }

A three semester calculus sequence should provide you with
the background needed for this class.  If you are uncertain
if you are ready for this class, please speak with me.

\subsubsection*{Grading }

Your course grade will be based on three examinations,
homework assignments, and a comprehensive final
examination.  The grading weights are

\vspace{0.1in}

\fbox{
\begin{minipage}{4.0in}
   Homework assignments \dotfill 35\% total \\
   Three examinations \dotfill 15\% each \\
   Final examination \dotfill 20\%  \phantom{each}
\end{minipage}
}

\vspace{0.1in}

\noindent Your home work grade will be based on the percent of the total available
points.


\subsubsection*{Homework}

Throughout the term you will be given problems that are to
be turned in for a grade.  The work you turn in
is expected to be {\em accurate, complete, concise, neat,}
and {\em well-organized.} The major steps of your
calculations should be explained using English sentences.
Homework assignments are due promptly at the start of class
the day they are due. {\em Late homework will not be
accepted.}

{\em Unless otherwise specified, all homework is to be
completed on your own.} If you are found in violation of
this policy, you will receive a zero on that
assignment.  If you are struggling with a homework  assignment, please ask
me for assistance. For everyone's benefit, I request
that you  ask me questions about assignments days, not hours, before it is due.

\subsubsection*{ Examinations}

Your three examinations and the final examination will
be in class and will be closed book and notes.  The
examinations are scheduled for {\em Friday 15 February},
{\em Friday 29 March}, and  {\em Friday 19 April}.
The final examination will be {\em Wednesday 8 May}
from 13:00--15:00.

\subsubsection*{Policies}

Regular class attendance and class participation are
required.  Please arrive to class on time. If have more than
five unexcused absences, your course grade will be reduced by
at least one half a letter grade.

If a exam conflicts with an official university sponsored
event, you will be allowed to take the exam early
provided you make arrangements at least four days before
the test.  Permission to take a test early for other
reasons will be given at my discretion.

If you are ill and are not able to take a test, you must call
or email me {\em before the test.} You will not be
allowed to take a test after I have returned the
test to the class.  Since I almost always return tests
the following class day, it is imperative that you make
arrangements will me without delay.

The final examination will be comprehensive.  It will be
given at the time determined by the University schedule.
(XXXX {\sc p.m.}) Except for extraordinary circumstances,
requests for taking the final examination at a different
time will not be granted.  All such requests must be made
before travel plans are made.

If your course average is $x$, your final grade
will be assigned according the function

\[ \mbox{grade}(x) =
  \left\{ \begin{array}{ll}

         F, & x \leq 60  \\

         D, &   60 < x \leq 65 \\

         D+, &  65 < x \leq 70 \\

         C,  &  70 < x \leq 75 \\

         C+, &  75 < x \leq 80 \\

         B,  &  80 < x \leq 86 \\

         B+, &  86 < x \leq 93 \\

         A,  &  93 < x

     \end{array}
    \right.
\]



Incomplete grades will be assigned according to University and
Departmental policies.



\subsubsection*{Office Hours and Getting in Touch with Me}



My office hours are listed on my office door and on this
syllabus.  You can also make an appointment for a specific
time to see me.  Additionally, anytime that I am in my
office, I will be glad to help you unless I am terribly
pressed for time.



\newpage



\subsubsection*{Course Calendar}



We will try to adhere to the following schedule, but we

will modify it if needed.



\vspace{0.1in}

\begin{center}



\begin{tabular}  {|l|l|l|}

\hline

Week  & Section(s) &  Topic(s) \\

\hline \hline

1     & \S1.1 -- 1.3  & Floating point numbers \\

2     & \S2.1 -- 2.2  & Linear equations \\

3     & \S2.3 -- 2.4  & Norms and condition

numbers \\

4     & \S3.1 -- 3.3  & Least squares \\

5     & \S4.1        & Eigenvalues  \\ \hline

6     & \S4.2        & Eigenvalues and first exam, Monday

16 Feb \\

7     & \S5.1        & Nonlinear equations \\

8     & \S5.2        & Nonlinear equations \\

9     & \S7.1        & Interpolation \\

10    & \S7.2        & Interpolation   \\ \hline

11    & \S8.1 -- 8.2  & Numerical integration and second exam,

Monday 30 March \\

12    & \S8.3 -- 8.4  & Adaptive and Gaussian Quadrature \\

13    & \S8.7, 9.1   & Finite differences and Differential

Equations \\

14   &  \S9.2 -- 9.4  & Differential Equations \\

15   &  \S12.1 -- 12.3  & Fast Fourier Transform, third

exam due Wednesday 6 May \\ \hline

     & \S1.1 -- 12.3    &  Final exam XXXXX  \\

\hline

\end{tabular}

\end{center}



\end{document}





